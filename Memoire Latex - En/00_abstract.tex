\addcontentsline{toc}{chapter}{Abstract}

\begin{abstract}

...and then there was sound. Without sound music could not have been born. Understanding sound is has been the quest of many throughout the years. Playing with timbre, dynamics, register, colors, naturality and every other aspect we can think of. During the second half of the 20th century composers and researchers starting wondering about what composes sounds. What is the components that make a sound of a violin differ from a sound of a flute when they play the same note. And thus spectralism was born as a path to the true understanding of sound.

Therefore this dissertation focuses, in its core, on the exploration of the nature of sound, and to be more precise, it's an investigation on spectral sound moprhing in real time. In this paper the process of spectral processing its going to be thourally analysed and a build from scratch Phase Vocoder is going to be presented. 

First is crucial to unvelop the mathematical notions behind the theory of spectral decomposition and re-synthesis. Therefore, section one deals with all the relevant math one will need to undestand the how to process spectra of sounds. Besides the mathematical formuli we set a musical context to an more musician-friendly approach.

Thereafter, we implement this mathematical context again in a musician-friendly programming environment such as the software MaxMSP. We investigate how the mathematical formuli interact with each other in a code setting. Step by step musical and programming explication build their way throught to the birth of a simple Phase Vocoder.

Finally, a combination of the knownledge we gained from the previous chapters and some basic musical programming principles we build a series of artistic implementations in MaxMSP that underline the artistic value of a phase vocoder.

One can expect this dissertation to be a musicians guide into spectral processing but also a fruitfull source of artistic examples and implementations. Its usefull for every musician or composer who seek to understand the logic behind the tools he frequently uses; anybody who wonders how to actually implement a Phase Vocoder in MaxMSP or any artists looking for ideas for a future project. It is to be expected from this research to envellop different uses of the Phase Vocoder and give access to further parametrisation.

\end{abstract}