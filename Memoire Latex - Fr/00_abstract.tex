\addcontentsline{toc}{chapter}{Abstract}

\begin{abstract}

Le compositeur Gérard Griséy dans son texte \textit{Écrits ou l'invention de la musique spectrale} disait: "L’architecture magnifie l’Espace disait Le Corbusier. Aujourd’hui comme jadis la musique transfigure le Temps" \footnote{Gérard Grisey, \textit{Écrits ou l'invention de la musique spectrale}, 2008  \nocite{Gr08}}. Par cette citation Gérard Grisey, démontre que sa musique, ou bien la musique spectrale, donne une autre dimension au temps. Il met ici en evidence l'importance de la temporalité dans la musique spectrale. En essayant de comprendre l'essentialité de la musique spectrale on est venu se confronter à la notion du spectre. Dès lors, les compositeurs spectrales ont expérimenté avec le timbre, la dynamique, le registre, les couleurs, le rythme et tous les autres aspects sonores auxquels on pourrait penser. Les compositeurs et les chercheurs se sont interrogés sur ce qui compose les sons. Quels sont les composants qui font qu’un son de violon est différent d’un son de flûte quand ils jouent la même note ?

%Par cette citation isu Gérard Griséy démontre le lien perpétuel entre des notions scintifiques et la musique. Mais la musique découle du son. Bien des hommes ont voulu comprendre l’essence du son.  Quels sont les composants qui font qu’un son de violon est différent d’un son de flûte quand ils jouent la même note ?

Cette question a défini la trajectoire de ce mémoire. L'objectif est d’explorer la nature du son et plus précisément l’étude de l'analyse spectrale en temps réel. Dans ce mémoire, le processus de traitement spectral va être analysé en profondeur et une construction d'une vocodeur de phase va être présenté pour le but d'une projection artistique.

Dans un premier temps, il est crucial d’envelopper les notions mathématiques qui sous-tendent la théorie de la décomposition spectrale et de la resynthèse. Par conséquent, la première section traite de toutes les mathématiques pertinentes dont on aura besoin pour décoder comment il est possible de traiter les spectres des sons. D’autre part, afin de faciliter la compréhension du cadre mathématique par des musiciens, nous intégrons un contexte musical. 

Ensuite, nous implémentons à nouveau ce contexte mathématique dans un environnement de programmation convivial pour les musiciens, tel que le logiciel MaxMSP. Nous étudions comment les formules mathématiques interagissent les unes avec les autres dans un code. Les explications musicales et de programmation se construisent pas à pas tout au long de la naissance d’un simple Vocoder de phase.

Enfin, en combinant les connaissances acquises au cours des chapitres précédents et quelques principes de programmation musicale de base, nous construisons une série d’applications artistiques dans MaxMSP afin de composer une pièce acousmatique qui va souligner la valeur artistique d’un vocodeur de phase.

On peut s’attendre à ce que ce mémoire soit un guide pour les musiciens sur le traitement spectral, mais également une source fructueuse d’exemples et d’applications artistiques. Ce mémoire  sera utile pour tout musicien ou compositeur qui cherche à comprendre la logique des outils qu’il utilise fréquemment mais aussi pour toute personne qui se demande comment implémenter réellement un Vocoder de phase dans MaxMSP ou tout artiste à la recherche d'idées pour un futur projet. Il faut s’attendre à ce que cette recherche englobe différentes utilisations du vocodeur de phase et donne d'accès à un paramétrage plus poussé.

\end{abstract}