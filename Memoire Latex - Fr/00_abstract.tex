\addcontentsline{toc}{chapter}{Abstract}

\begin{abstract}

... et puis il y avait du son. Sans le son, la musique n'aurait pu naître. Comprendre le son a été la quête de beaucoup de gens au fil des ans. Jouer avec le timbre, la dynamique, le registre, les couleurs, la naturalité et tous les autres aspects auxquels nous pouvons penser. Au cours de la seconde moitié du XXe siècle, les compositeurs et les chercheurs se sont interrogés sur ce qui compose les sons. Quels sont les composants qui font qu'un son de violon est différent d'un son de flûte quand ils jouent la même note? C'est ainsi que le spectralisme est né comme voie menant à la véritable compréhension du son.

Par conséquent, cette thèse se concentre essentiellement sur l'exploration de la nature du son et, pour être plus précis, sur une étude du son spectral en temps réel. Dans cet article, le processus de traitement spectral va être analysé en profondeur et un build de phase Vocoder va être présenté.

Premièrement, il est crucial d’envelopper les notions mathématiques qui sous-tendent la théorie de la décomposition spectrale et de la resynthèse. Par conséquent, la première section traite de toutes les mathématiques pertinentes dont on aura besoin pour comprendre comment traiter les spectres des sons. Outre les formules mathématiques, nous plaçons un contexte musical dans une approche plus conviviale pour les musiciens.

Ensuite, nous implémentons à nouveau ce contexte mathématique dans un environnement de programmation convivial pour les musiciens, tel que le logiciel MaxMSP. Nous étudions comment les formules mathématiques interagissent les unes avec les autres dans un code. Des explications musicales et de programmation pas à pas se construisent tout au long de la naissance d’un simple Vocoder de phase.

Enfin, en combinant les connaissances acquises au cours des chapitres précédents et quelques principes de programmation musicale de base, nous construisons une série d’applications artistiques dans MaxMSP qui soulignent la valeur artistique d’un vocodeur de phase.

On peut s’attendre à ce que cette thèse soit un guide des musiciens pour le traitement spectral, mais également une source fructueuse d’exemples et d’applications artistiques. Son utilité pour tout musicien ou compositeur qui cherche à comprendre la logique des outils qu’il utilise fréquemment; Quiconque se demande comment implémenter réellement un Vocoder de phase dans MaxMSP ou tout artiste à la recherche d'idées pour un futur projet. Il faut s’attendre à ce que ces recherches englobent différentes utilisations du vocodeur de phase et donnent accès à un paramétrage plus poussé.

\end{abstract}