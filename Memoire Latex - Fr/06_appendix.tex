

\label{ch:Appendix}

	\section{FFT Cooley - Tukey algorithm}\footnote{Le code est recuperé du lien \href{https://www.algorithm-archive.org/contents/cooley_tukey/cooley_tukey.html}{www.algorithm-archive.org} distribué sous une licence \textit{MIT}}
	\label{Cooley-Tukey_Code} 

	\begin{lstlisting}[language=C, caption= Algorithme de Cooley-Tukey avec des diagrammes papillon]
		using FFTW
		#simple DFT function
		function DFT(x)s
		    N = length(x)
		    # We want two vectors here for real space (n) and frequency space (k)
		    n = 0:N-1
		    k = n'
		    transform_matrix = exp.(-2im*pi*n*k/N)
		    return transform_matrix*x
		end
		# Implementing the Cooley-Tukey Algorithm
		function cooley_tukey(x)
		    N = length(x)
		    if (N > 2)
		        x_odd = cooley_tukey(x[1:2:N])
		        x_even = cooley_tukey(x[2:2:N])
		    else
		        x_odd = x[1]
		        x_even = x[2]
		    end
		    n = 0:N-1
		    half = div(N,2)
		    factor = exp.(-2im*pi*n/N)
		    return vcat(x_odd .+ x_even .* factor[1:half],
		                x_odd .- x_even .* factor[1:half])
		end
		function bitreverse(a::Array)
		    # First, we need to find the necessary number of bits
		    digits = convert(Int,ceil(log2(length(a))))

		    indices = [i for i = 0:length(a)-1]

		    bit_indices = []
		    for i = 1:length(indices)
		        push!(bit_indices, bitstring(indices[i]))
		    end
		    # Now stripping the unnecessary numbers
		    for i = 1:length(bit_indices)
		        bit_indices[i] = bit_indices[i][end-digits:end]
		    end
		    # Flipping the bits
		    for i =1:length(bit_indices)
		        bit_indices[i] = reverse(bit_indices[i])
		    end
		    # Replacing indices
		    for i = 1:length(indices)
		        indices[i] = 0
		        for j = 1:digits
		            indices[i] += 2^(j-1) * parse(Int, string(bit_indices[i][end-j]))
		        end
		       indices[i] += 1
		    end
		    b = [float(i) for i = 1:length(a)]
		    for i = 1:length(indices)
		        b[i] = a[indices[i]]
		    end
		    return b
		end
		function iterative_cooley_tukey(x)
		    N = length(x)
		    logN = convert(Int,ceil(log2(length(x))))
		    bnum = div(N,2)
		    stride = 0;
		    x = bitreverse(x)
		    z = [Complex(x[i]) for i = 1:length(x)]
		    for i = 1:logN
		       stride = div(N, bnum)
		       for j = 0:bnum-1
		           start_index = j*stride + 1
		           y = butterfly(z[start_index:start_index + stride - 1])
		           for k = 1:length(y)
		               z[start_index+k-1] = y[k]
		           end
		       end
		       bnum = div(bnum,2)
		    end
		    return z
		end
		function butterfly(x)
		    N = length(x)
		    half = div(N,2)
		    n = [i for i = 0:N-1]
		    half = div(N,2)
		    factor = exp.(-2im*pi*n/N)

		    y = [0 + 0.0im for i = 1:length(x)]

		    for i = 1:half
		        y[i] = x[i] + x[half+i]*factor[i]
		        y[half+i] = x[i] - x[half+i]*factor[i]
		    end
		    return y
		end
		function approx(x, y)
		    val = true
		    for i = 1:length(x)
		        if (abs(x[i]) - abs(y[i]) > 1e-5)
		            val = false
		        end
		    end
		    println(val)
		end
		function main()
		    x = rand(128)
		    y = cooley_tukey(x)
		    z = iterative_cooley_tukey(x)
		    w = fft(x)
		    approx(y, w)
		    approx(z, w)
		end
		main()
	\end{lstlisting}

\begin{comment}

\section{Analyse fonctionelle et analyse de Fourier}

	La STFT est une version locale de la transformation de Fourier. Normalement, pour effectuer la transformation de Fourier, on part d'une fonction $f(x)$ qui appartient à l'espace Lebesgue $L^{2} (\mathbb{R}^k)$ soit à $L^{1} (\mathbb{R}^k)$ et on obtient une fonction $\hat{f}(\omega)$ qui appartient respectivement à $L^{2} (\mathbb{R}^k)$ ou à $C_{0}(\mathbb{R}^{k})$\footnote{L'espace des fonctions continues qui tendent vers $0$ à l'infini}. L'espace $L^2$ est l'espace des fonctions de carré intégrable.

	Si la transformée de Fourier de {{mvar|f}}, notée <math>\hat f</math>, est elle-même une [[fonction intégrable]], la formule dite de transformation de Fourier inverse, opération notée <math>\mathcal F^{-1}</math>, et appliquée à <math>\hat f</math>, permet (sous conditions appropriées) de retrouver {{mvar|f}} à partir des données fréquentielles :
	$$ 
	f(x)= F^{-1} (\hat f)(x)={1 \over 2\pi}\, \int_{-\infty}^{+\infty} \hat f(\xi)\, e^{+{\rm i}\xi x} \; d\xi \iff \hat f(\xi) = \int_{-\infty}^{+\infty} f(x) \mathrm e^{-{\rm i}\xi x}\,\mathrm 
	$$

	Pour résumer, on a bien un isomorphisme isométrique sur la transformation de Fourier de $L^2(\mathbb{R}^k)$ en $L^2(\mathbb{R}^k)$, ce qui de manière symbolique est traduit par \footnote{Karlheinz Gröchenig, \textit{Foundations of Time-Frequency Analysis}, 2001. \nocite{Grochenig}} :
	$$
	\begin{matrix}
	\mathcal{F} : & L^2(\mathbb{R}^k) & \rightarrow & L^2(\mathbb{R}^k) \\
	& f & \mapsto & \hat{f}
	\end{matrix}
	$$
	et
	$$ \mathcal{F} \circ \mathcal{F}^{-1} = \mathcal{F}^{-1} \circ \mathcal{F} = Id $$

	Pour réaliser une STFT on doit definir une fonction $g$ qui servira comme fenêtrage. Géneriquement, on utilise des fonctions avec une forme décroissance vers l'infini (Filtres Gabor). On examine, donc, si $g\in L^{p}(\mathbb{R}^{k})$ alors on aura que $f\in L^{p'} (\mathbb{R}^k)$ admet une STFT pour $f$ possible.

	Pour appliquer la STFT à une fonction il faut que le résultat soit bien défini. Cela introduit le problème de choisir un domaine de définition pour la STFT.
	Dans cette définition de la STFT on choisi comme domaines pour $f$ et pour $g$ respectivement $L^{p'} (\mathbb{R}^k) $ et $L^p (\mathbb{R}^k)$. Bien évidement la STFT peut être étendue à d'autres espaces (ex. notament des espaces de distributions), mais dans ce cas on la regarde sous l'oeil des espaces Lebesgue.

	Soient $p, p'$ (i.e. $\frac{1}{p} + \frac{1}{p'} = 1$). Alors, $\forall f \in L^{p'}(\mathbb{R}^{k})$, $\forall g\in L^{p}(\mathbb{R}^{k})$, $V_{g}f$ est bien définie, i.e. $\forall(t, \omega)\in \mathbb{R}^{k} \times \mathbb{R}^{k}$, $V_{g}f(t, \omega) \in \mathbb{C}$.

	Soient $f \in L^{p'}(\mathbb{R}^{k})$, $g\in L^{p}(\mathbb{R}^{k})$.

	Pour tout $t \in \mathbb{R}^{k}$, l'opérateur de traslation du vecteur $t$, qu'on note $T_{t}$, est une isométrie de $L^{p}(\mathbb{R}^{k})$ en soi même. On a donc $g\in L^{p}(\mathbb{R}^{k}) \Rightarrow \overline{g}\in L^{p}(\mathbb{R}^{k}) \Rightarrow T_{t}\overline{g} \in L^{p}(\mathbb{R}^{k})$.

	On a donc que $f\cdot T_{t}\overline{g} \in L^{1}(\mathbb{R}^{k})$ car, grâce à l'inégalité de Hölder \footnote{L.P. Kuptsov, \textit{Hölder inequality}, 1994, \nocite{Holder}}.

	$$ \|f\cdot T_{t}\overline{g}\|_{1} \leq  \|f\|_{p'} \|T_{t}\overline{g}\|_{p} < \infty$$

	On sait donc que pour tout $t\in\mathbb{R}^{k}$ la transformation de Fourier de $f\cdot T_{t}\overline{g}$ est bien définie grâce au Lemme de Riemann-Lebesgue et donc que $\forall \omega \in \mathbb{R}^{k}$, $V_{g}f(t, \omega) = \mathcal{F}[f\cdot T_{t}\overline{g}](\omega) \in \mathbb{C}$.

	On a prouvé qu'une fonction $f$ est isomorphe à sa transformation de Fourier dans des espaces Lebesgue et, de plus, la STFT est bien défini dans $L^p$.

\end{comment}