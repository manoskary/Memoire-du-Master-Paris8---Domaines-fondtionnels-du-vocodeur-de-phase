\usepackage{graphicx}
\usepackage{verbatim}
\usepackage{latexsym}
\usepackage{mathchars}
\usepackage{setspace}
\usepackage[french]{babel}
\selectlanguage{french}
\usepackage[T1]{fontenc}
\usepackage[utf8]{inputenc}
\usepackage{natbib}
\usepackage{hyperref}
\usepackage{comment}
\usepackage{pgfplots}
\usepackage{listings}
\usepackage{color}
\usepackage{amsmath}
\usepackage{setspace}
\usepackage{amssymb}
\usepackage{csquotes}
\usepackage{mathrsfs}
\usepackage{caption}
\usepackage{subcaption}
\usepackage{wrapfig}

\setlength{\parskip}{1em}  % a little space before a \par
\setlength{\parindent}{5pt}	      % don't indent first lines of paragraphs

%\setlength{\parskip}{2.5em}  % a little space before a \par
%\setlength{\parindent}{2pt}	      % don't indent first lines of paragraphs
%UHEAD.STY  If this is included after \documentstyle{report}, it adds
% an underlined heading style to the LaTeX report style.
% \pagestyle{uheadings} will put underlined headings at the top
% of each page. The right page headings are the Chapter titles and
% the left page titles are supplied by \def\lefthead{text}.

% Ted Shapin, Dec. 17, 1986\\

\makeatletter
\def\chapapp2{Chapter}

\def\appendix{\par
 \setcounter{chapter}{0}
 \setcounter{section}{0}
 \def\chapapp2{Appendix}
 \def\@chapapp{Appendix}
 \def\thechapter{\Alph{chapter}}}

\def\ps@uheadings{\let\@mkboth\markboth
% modifications
\def\@oddhead{\protect\underline{\protect\makebox[\textwidth][l]
		{\sl\rightmark\hfill\rm\thepage}}}
\def\@oddfoot{}
\def\@evenfoot{}
\def\@evenhead{\protect\underline{\protect\makebox[\textwidth][l]
		{\rm\thepage\hfill\sl\leftmark}}}
% end of modifications
\def\chaptermark##1{\markboth {\ifnum \c@secnumdepth >\m@ne
 \chapapp2\ \thechapter. \ \fi ##1}{}}%
\def\sectionmark##1{\markright {\ifnum \c@secnumdepth >\z@
   \thesection. \ \fi ##1}}}
\makeatother

\def\Beginboxit
   {\par
    \vbox\bgroup
	   \hrule
	   \hbox\bgroup
		  \vrule \kern1.2pt %
		  \vbox\bgroup\kern1.2pt
   }

\def\Endboxit{%
			      \kern1.2pt
		       \egroup
		  \kern1.2pt\vrule
		\egroup
	   \hrule
	 \egroup
   }	

\newenvironment{boxit}{\Beginboxit}{\Endboxit}
\newenvironment{boxit*}{\Beginboxit\hbox to\hsize{}}{\Endboxit}
\pagestyle{empty}

\setlength{\parskip}{2ex plus 0.5ex minus 0.2ex}
\setlength{\parindent}{0pt}

\makeatletter  %to avoid error messages generated by "\@". Makes Latex treat "@" like a letter

\linespread{1.5}
\def\submitdate#1{\gdef\@submitdate{#1}}

\def\maketitle{
  \begin{titlepage}{
    %\linespread{1.5}
    \Large Université Paris8 \\
    %\linebreak
    Théories et Pratiques de la musique \\
    %\linebreak
    Composition et Réalisation
    \rm
    \vskip 3in
    \Large \bf \@title \par
  }
  \vskip 0.3in
  \par
  {\Large \@author}
  \vskip 4in
  \par
  
 % \linebreak
  
  \linebreak
   \@submitdate
  \vfil
  \end{titlepage}
}

\def\titlepage{
  \newpage
  \centering
  \linespread{1}
  \normalsize
  \vbox to \vsize\bgroup\vbox to 9in\bgroup
}
\def\endtitlepage{
  \par
  \kern 0pt
  \egroup
  \vss
  \egroup
  \cleardoublepage
}

\def\abstract{
  \begin{center}{
    \large\bf Resumé}
  \end{center}
  \small
  %\def\baselinestretch{1.5}
  \linespread{1.5}
  \normalsize
}
\def\endabstract{
  \par
}

\newenvironment{acknowledgements}{
  \cleardoublepage
  \begin{center}{
    \large \bf Acknowledgements}
  \end{center}
  \small
  \linespread{1.5}
  \normalsize
}{\cleardoublepage}
\def\endacknowledgements{
  \par
}

\newenvironment{dedication}{
  \cleardoublepage
  \begin{center}{
    \large \bf Dedication}
  \end{center}
  \small
  \linespread{1.5}
  \normalsize
}{\cleardoublepage}
\def\enddedication{
  \par
}

\def\preface{
    \pagenumbering{roman}
    \pagestyle{plain}
    \doublespacing
}

\def\body{
    \cleardoublepage    
    \pagestyle{uheadings}
    \tableofcontents
    \pagestyle{plain}
    \cleardoublepage
    \pagestyle{uheadings}
%    \listoftables
%   \pagestyle{plain}
%    \cleardoublepage
%    \pagestyle{uheadings}
    \listoffigures
    \pagestyle{plain}
    \cleardoublepage
    \pagestyle{uheadings}
    \pagenumbering{arabic}
    \doublespacing
}

\makeatother  %to avoid error messages generated by "\@". Makes Latex treat "@" like a letter

\pgfplotsset{compat=1.16}

\newcommand{\ipc}{{\sf ipc}}




\newcommand{\op}[1]{\mathrm{#1}}
\newcommand{\s}[1]{\ensuremath{\mathcal #1}}

%Personalization for code display
 
\definecolor{codeblue}{rgb}{0,0.1,0.9}
\definecolor{codegray}{rgb}{0.5,0.5,0.5}
\definecolor{codepurple}{rgb}{0.58,0,0.82}
\definecolor{backcolour}{rgb}{0.91,0.9,0.89}
 
\lstdefinestyle{mystyle}{
    backgroundcolor=\color{backcolour},   
    commentstyle=\color{codeblue},
    keywordstyle=\color{magenta},
    numberstyle=\tiny\color{codegray},
    stringstyle=\color{codepurple},
    basicstyle=\footnotesize,
    breakatwhitespace=false,         
    breaklines=true,                 
    captionpos=b,                    
    keepspaces=true,                 
    numbers=left,                    
    numbersep=5pt,                  
    showspaces=false,                
    showstringspaces=false,
    showtabs=false,                  
    tabsize=2
}
 
\lstset{style=mystyle}

% Properly styled differentiation and integration operators
\newcommand{\diff}[1]{\mathrm{\frac{d}{d\mathit{#1}}}}
\newcommand{\diffII}[1]{\mathrm{\frac{d^2}{d\mathit{#1}^2}}}
\newcommand{\intg}[4]{\int_{#3}^{#4} #1 \, \mathrm{d}#2}
\newcommand{\intgd}[4]{\int\!\!\!\!\int_{#4} #1 \, \mathrm{d}#2 \, \mathrm{d}#3}

% Large () brackets on different lines of an eqnarray environment
\newcommand{\Leftbrace}[1]{\left(\raisebox{0mm}[#1][#1]{}\right.}
\newcommand{\Rightbrace}[1]{\left.\raisebox{0mm}[#1][#1]{}\right)}

% Funky symobols for footnotes
\newcommand{\symbolfootnote}{\renewcommand{\thefootnote}{\fnsymbol{footnote}}}
% now add \symbolfootnote to the beginning of the document...

\newcommand{\normallinespacing}{\renewcommand{\baselinestretch}{1.5} \normalsize}
\newcommand{\mediumlinespacing}{\renewcommand{\baselinestretch}{1.2} \normalsize}
\newcommand{\narrowlinespacing}{\renewcommand{\baselinestretch}{1.0} \normalsize}
\newcommand{\bump}{\noalign{\vspace*{\doublerulesep}}}
\newcommand{\cell}{\multicolumn{1}{}{}}
\newcommand{\spann}{\mbox{span}}
\newcommand{\diagg}{\mbox{diag}}
\newcommand{\modd}{\mbox{mod}}
\newcommand{\minn}{\mbox{min}}
\newcommand{\andd}{\mbox{and}}
\newcommand{\forr}{\mbox{for}}
\newcommand{\EE}{\mbox{E}}

\newcommand{\deff}{\stackrel{\mathrm{def}}{=}}
\newcommand{\syncc}{~\stackrel{\textstyle \rhd\kern-0.57em\lhd}{\scriptstyle L}~}

\def\coop{\mbox{\large $\rhd\!\!\!\lhd$}}
\newcommand{\sync}[1]{\raisebox{-1.0ex}{$\;\stackrel{\coop}{\scriptscriptstyle
#1}\,$}}

\newtheorem{definition}{Definition}[chapter]
\newtheorem{theorem}{Theorem}[chapter]

\newcommand{\Figref}[1]{Figure~\ref{#1}}
\newcommand{\fig}[3]{
 \begin{figure}[!ht]
 \begin{center}
 \scalebox{#3}{\includegraphics{figs/#1.ps}}
 \vspace{-0.1in}
 \caption[ ]{\label{#1} #2}
 \end{center}
 \end{figure}
}

\newcommand{\figtwo}[8]{
 \begin{figure}
 \parbox[b]{#4 \textwidth}{
 \begin{center}
 \scalebox{#3}{\includegraphics{figs/#1.ps}}
 \vspace{-0.1in}
 \caption{\label{#1}#2}
 \end{center}
 }
 \hfill
 \parbox[b]{#8 \textwidth}{
 \begin{center}
 \scalebox{#7}{\includegraphics{figs/#5.ps}}
 \vspace{-0.1in}
 \caption{\label{#5}#6}
 \end{center}
 }
 \end{figure}
}