\label{ch:conclusions}

\section{Résumé de la recherche}

%Resumer tout qui était fait.
Dans ce mémoire, nous avons parcouru des notions mathématiques de l'analyse spectrale en se focalisant sur le vocodeur de phase. Nous avons cité toute information nécessaire pour qu'un amateur se plonge dans le domaine du processus spectral du signal. De plus, nous avons détaillé le processus pour créer un vocodeur de phase sur Max permettant une connaissance plus profonde parmi l'application. En outre, nous avons proposé plusieurs modifications du code dans un contexte artistique et nous avons composé une pièce acousmatique en utilisant ces modifications.

En d'autres termes, cette recherche sert à des fins documentaires ainsi qu'épistémologique. Notre but est d'informer les musiciens et les compositeurs des notions complexes. Ce travail me sera aussi enrichissant, car il m'a permis d'approfondir mes connaissances dans ce domaine. Le spectre est une terminologie difficile à comprendre pour les musiciens et un point de vue plus musical va faciliter sa compréhension. En approfondissant sur le terrain du morphing, on découvre de nouvelles manières pour manipuler cet effet et on applique notre critique. Cette problématique, constitue le but final de la recherche.

\section{Applications}

Dans le second chapitre, nous avons parcouru plusieurs applications du vocodeur de phase mais les capacités de cet outil sont quasiment infinies. Pour donner une idée de ces capacités, le vocodeur de phase peut être utilisé pour analyser la voix et la transformer en texte. Les sélections et les fréquences peuvent être déduites de certaines voyelles et consonnes, ce qui permet de prédire les lettres exactes utilisées et de produire la forme écrite du son\footnote{Axel Roebel, \textit{Shape-invariant speech transformation with the phase vocoder}, 2010.\nocite{roebel2010shape}}.

Le vocodeur de phase qui est appliqué exclusivement sur la voix humaine peut être également modifié pour transformer un certain type de voix. Par exemple une voix d'homme peut être transformée en une voix féminine ou bien tout autre type de voix humaine. Cette opération est construite sur une base des profils spectraux de différents types des voix qui permette d'interpoler entre ces profils spectrales comme un morphing sous paramètres \footnote{Axel Roebel, \textit{A new approach to transient processing in the phase vocoder}, 2003. \nocite{roebel:hal-01161124}}

Une amélioration du vocodeur de phase consiste à appliquer des ondelettes. En lieu d'appliquer la transformation simple de Fourier, on peut remplacer l'exponentiel complexe de Fourier par une fonction d'ondelette. Cette fonction est équivalente de l'application de plusieurs transformées de Fourier à différentes tailles de fenêtres permettant de capter simultanément plus de détail fréquentiel et ponctuel (attaque)\footnote{{Beltr{\'a}n, Jos{\'e} R and Beltr{\'a}n, Fernando}, \textit{Additive synthesis based on the continuous wavelet transform}, 2003.\nocite{beltran2003additive}}. 

Les outils spectraux et spécialement ceux d'analyse-synthèse sont fréquemment utilisés par les compositeurs parmi les logiciels DAW tels que Ableton Live, Cubase, Reaper, etc. Une des  bibliothèques fréquemment utilisée est TRAX fait par \textit{IRCAM Tools} \footnote{\href{https://www.flux.audio/project/ircam-trax-v3/}{https://www.flux.audio/project/ircam-trax-v3/}}.

\section{Discussion}

Le vocodeur de phase est aujourd'hui plus qu'un simple outil d'analyse spectrale. Des méthodes ont été développées pour la prévision de manière stochastique sur une re-synthèse optimale des ondes sonores.

Ces méthodes considèrent des formules probabilistes pour reconstruire des signaux afin d’obtenir une meilleure technique de prédiction de la fréquence. Pour plus d'informations à ce sujet, le livre de Roads and al. \footnote{Curtis Roads, Stephen Travis Pope, Aldo Piccialli, Giovanni De Poli, \textit{Musical Signal Processing}, 1997 \nocite{Roads97}} était plutôt éclairant.

\section{Recherche pour l'avenir}

On peut étendre cette recherche sur les processus spectraux en utilisant les principes tels que Deep Learning. Spécifiquement, j'envisage une approche sur le Morphing spectrale en temps réel avec de l'apprentissage pour automatiser la paramétrisation du vocodeur de phase correspondant \footnote{Jesse Engel. Making a neural synthesizer instrument, 2008.}. 

Je suis particulièrement intéressé par le fait qu’un vocodeur de phase puisse détecter les fréquences dominantes d'un son. Ce fait pourrait essentiellement servir afin de développer une méthode intelligente pour préciser les différentes notes contenue dans un son polyphonique \footnote{Polyphonie : Assemblage de voix ou d'instruments, sans préjuger de leur nature}. L'extraction des fréquences dominantes de chaque voix en forme de notes pourrait assister à l'analyse automatique de la musique à partir des fichiers sonores et, par suite, transformer le son à une forme symbolique. Dans une prochaine recherche, je souhaiterais aborder le domaine de l'analyse Topologique de la musique. C’est une interprétation de la musique symbolique comme des structures topologiques bidimensionnelles, c'est-à-dire des graphes orientés. À partir de ces graphes on pourra extraire des informations du style musical ou de la structure harmonique d'une pièce musicale. La théorie de graphes contient plusieurs méthodes spectrales telles que la transformation Fourier des graphes et toute une infrastructure mathématique basée sur les concepts investigués dans ce mémoire.

Une  des pistes alternatives, également intéressante à étudier pour ma part,, serait l’utilisation des VAE (Variational Autoencoders) pour identifier la distribution de probabilité d’un ensemble de pièces musicales. Les VAE sont des auto-encodeurs qui paramètrent la distribution de probabilité des variables latentes à l'aide d'un réseau neuronal et maximisent la probabilité que la distribution de données soit soumise à des contraintes de la distribution de variables latentes. Des expériences numériques montrent qu’il est capable d’identifier cette distribution de données dans la mesure où l’échantillonnage à partir de la distribution de variables latentes génère des points de données réalistes. J'aspire à utiliser cette représentation pour produire une nouvelle musique en combinant d'autres pièces musicales dans l'intégration VAE.

Je crois que le processus d'analyse musicale sur l'harmonie et la mélodie, associé à un apprentissage non supervisé, permet de passer à de meilleurs modèles de prédiction musicale, de variation et de créativité. Pour atteindre cet objectif, je visualise la première formalisation de modélisation algébrique sur des données verticales (accords) et linéaires (mélodies) telles que les approches de la théorie de la musique transformationnelle. Avant de construire une base de données de styles et d’appliquer un algorithme de prédiction, j’estime que, grâce à VAE, cela peut être avancé et produire des résultats révolutionnaires.
Jusqu'à présent, VAE présentait une application limitée aux données séquentielles et les modèles existants de VAE récurrents rencontraient des difficultés pour modéliser des séquences avec une structure à long terme. Pour résoudre ce problème, je propose l’utilisation d’un décodeur hiérarchique, qui génère d’abord des intégrations pour les sous-séquences de l’entrée, puis utilise ces intégrations pour générer chaque sous-séquence de manière indépendante. Il est également possible de modéliser de la musique polyphonique multi-piste en tant que vecteurs dans un espace latent via le conditionnement d'accords. En effet celui-ci permet une compréhension plus profonde de la mélodie tout en maintenant une harmonie fixe, et permet en outre de modifier les accords tout en maintenant le style musical. Cette approche a été présentée pour la première fois dans le projet Magenta de Google AI et présente, à mon avis, un potentiel important et pourrait progresser dans la génération simultanée d'accords et de mélodies.
